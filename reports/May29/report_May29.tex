\documentclass{beamer}

% For more themes, color themes and font themes, see:
% http://deic.uab.es/~iblanes/beamer_gallery/index_by_theme.html
%
\mode<presentation>
{
  \usetheme{Madrid}       % or try default, Darmstadt, Warsaw, ...
  \usecolortheme{beaver} % or try albatross, beaver, crane, ...
  \usefonttheme{serif}    % or try default, structurebold, ...
  \setbeamertemplate{navigation symbols}{}
  \setbeamertemplate{caption}[numbered]
} 

\usepackage[english]{babel}
\usepackage[utf8x]{inputenc}
\usepackage{amsmath} % macros ams
\usepackage{amsfonts} % fonts ams
\usepackage{amssymb} % symboles ams
\usepackage{subfigure}

% On Overleaf, these lines give you sharper preview images.
% You might want to `comment them out before you export, though.
\usepackage{pgfpages}
\pgfpagesuselayout{resize to}[%
  physical paper width=8in, physical paper height=6in]

% Here's where the presentation starts, with the info for the title slide
\title[Shock wave refraction]{Lagrangian particles evolution throught expansion fan, from Yann's work}
\author{M. Dutreuilh}
\institute{Tsinghua University}
\date{\today}

\begin{document}

\begin{frame}
  \titlepage
\end{frame}

% These three lines create an automatically generated table of contents.
% \begin{frame}{Outline}
%   \tableofcontents
% \end{frame}

\section{1 - Bibliography}

\begin{frame}{1 - Bibliography}

\begin{itemize}
  \item Groove, Menikoff. Anomalous reflection of a shock wave at a fluid interface (1990)
  \item Haas, Sturtevant. Interaction of weak shock waves with cydrical and spherical gas inhomogeneities (1987)
\end{itemize}

Given by my teacher of compressible flows at ENSTA : has to be read carefully \dots{}

\end{frame}

\section{Evolution of a lagragian particle in the $H_2-O_2$ phase}
\begin{frame}{Evolution of a lagragian particle in the $H_2-O_2$ phase}

\begin{figure}
    \centering
    \includegraphics[scale=0.5]{logo.jpg}
\end{figure}

\end{frame}

\section{3 - Boring procedures}

\begin{frame}{3 - Boring procedures}

\begin{itemize}
\item Internship contract is not signed by everyone yet 
\item Officially from May, 21 to 24, July, 35 hours a week
\end{itemize}

\end{frame}

\section{4 - What's next?}

\begin{frame}{4 - What's next?}

\begin{itemize}
\item Bibliography reading, starting with the 2 new articles + book Shock Refraction Phenomena + re-read articles related with polar plots
\item Understand well what differentiates regular refraction from irregular refraction 
\item Ask Yann for explanations about what's wrong in his programs
\item Fix programs
\end{itemize}

\end{frame}

\end{document}